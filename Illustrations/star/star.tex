\documentclass{article}
\usepackage{mathptmx}
\usepackage{tikz}
\usepackage{color}
\usepackage{amsmath}
\DeclareSymbolFont{symbolsb}{OMS}{cmsy}{m}{n}
\SetSymbolFont{symbolsb}{bold}{OMS}{cmsy}{b}{n}
\DeclareSymbolFontAlphabet{\mathcal}{symbolsb}
\usepackage[active,pdftex,tightpage]{preview}
\PreviewEnvironment[]{tikzpicture}
\PreviewEnvironment[]{pgfpicture}
\usetikzlibrary{ decorations.markings, calc, fadings,
decorations.pathreplacing, patterns, decorations.pathmorphing, positioning}
\usetikzlibrary{calc}


\begin{document}

\newcommand{\AxisRotator}[1][rotate=0] {%
  \tikz[decoration={
    markings,
    mark=at position 1 with {\arrow{latex}}}]\draw[x = .5em, y = 2.75em, line width = .2ex,#1,postaction=decorate] (0,0)  arc (-150:150:.45 and .5) -- ++(-95:2pt);%                                                   
  }

\begin{figure}[ht]
\centering

\begin{tikzpicture}[scale=1.75] 
	% Define some things

	\def\cos{1.0}
	\def\sin{0.0}
	\def\outer{2.5}
	\def\rcore{0.7}
      \def\rmag{1.0}
      \def\frac{0.8}


	% Colors
	\colorlet{anglecolor}{blue!80!black}
	\colorlet{sincolor}{red}
	\colorlet{tancolor}{orange!80!black}
	\colorlet{coscolor}{blue}

	% Styles
	\tikzstyle{axes}=[]
	\tikzstyle{important line}=[very thick]
	\tikzstyle{information text}=[rounded corners,fill=red!10,inner sep=1ex]

\node[inner sep=0pt] (GW1) at (0,-3.1)
    {\includegraphics[trim= 0mm 0mm 0mm 0mm, width=1.1\textwidth]{../GWs/GW1}};





\begin{scope}[yshift=-3cm]

 % Draw the back beams
%\begin{scope}[xshift=-2.35cm,,yshift=40,scale=0.45,rotate=-45]
%	\draw[dashed,color=gray] (0,0) arc (-90:90:0.5 and 1.5);% right half of the left ellipse
%	\draw[semithick] (0,0) -- (4,1);% bottom line
%	\draw[semithick] (0,3) -- (4,2);% top line
%	\draw[semithick] (0,0) arc (270:90:0.5 and 1.5);% left half of the left ellipse
%	\draw[semithick] (4,1.5) ellipse (0.166 and 0.5);% right ellipse
%	\fill[color=cyan,opacity=0.5] (0,0) arc (270:90:0.5 and 1.5) -- (6,1.5) -- cycle;
%\end{scope}

\pgfdeclareradialshading{ballshading}{\pgfpoint{30bp}{-20bp}}
 {color(0bp)=(white); 
 color(9bp)=(cyan!25!white);
 color(18bp)=(cyan!90!black); 
 color(25bp)=(cyan!60!black); 
 color(50bp)=(black)}

% Draw the shaded star
%
%\shade [shading=ballshading, opacity=0.9] (0,0) circle [radius=\rmag]; 
\filldraw[ball color=brown,opacity=1.0] (0.1,0.2) ellipse [x radius = \rcore, y radius= 0.99*\rcore];
%\shade [shading=ballshading, opacity=0.2] (0,0) circle [radius=\rmag]; 

%\filldraw[ball color=green,opacity=0.3] (0,0) ellipse (2.0 and 2.0);

% Draw rotation
 %\draw[color=black] (0.0,0.0) node at (1.5*\sin,1.5*\cos) {\AxisRotator[rotate = 90]};

% Define omega to have unit magnitude and lie at 30 degrees to the z axis
%
%\draw[color=black,very thick, line cap=round] (\frac * \rmag*\sin,\frac * \rmag*\cos) -- (\outer*\sin,\outer*\cos) ;
%\draw[color=black, very thick, opacity=0.3, line cap=round] (\frac * \rcore * \sin,\frac * \rcore*\cos)  -- (\rmag*\sin,\rmag*\cos) ;
%\draw[color=black, very thick, opacity=0.3, line cap=round] (-\rcore * \sin,-\rcore*\cos)  -- (-\rmag*\sin,-\rmag*\cos) ;
%\draw[color=black,very thick, opacity=0.3, line cap=butt] (-\rmag*\sin,-\rmag*\cos)  -- (-\outer*\sin,-\outer*\cos) ;
%
%%
%\begin{scope}[xshift=2.7cm,,yshift=-37,scale=0.5,rotate=140]
%	\draw[dashed] (4,1.0) arc (270:90:0.166 and 0.5);
%
%	% Fill in the colour
%
%	
%	\fill[color=cyan,opacity=0.4] (0,0.0) -- (0,3.0) -- (4,2.0) -- (4,1.0)  -- cycle;
%	\fill[color=cyan!70!white,opacity=1.0]  (0,1.5) ellipse (0.5 and 1.5)  -- cycle;
%
%	\draw[semithick,color=black] (0,0) arc (-90:90:0.5 and 1.5);% right half of the left ellipse
%	\draw[semithick] (0,0) -- (4,1);% bottom line
%	\draw[semithick] (0,3) -- (4,2);% top line
%	\draw[semithick] (0,0) arc (270:90:0.5 and 1.5);% left half of the left ellipse
%	
%	\draw[semithick] (4,1.0) arc (270:450:0.166 and 0.5);% right ellipse
%	\fill[color=cyan!30!white,opacity=1.0]  (4,1.5) ellipse (0.166 and .5) -- cycle;
%
%\end{scope}

% Annotations
%
%\draw[] (0, 0); \node[]{\large Crust};
%\draw[<-, thick] (\rmag*0.6, \rmag*0.6) -- (0.9,1.5) node[right]{\large Magnetosphere};
%\draw[<-, thick] (1.7, -1.3) -- (2.8,2.2);
%\draw[<-, thick] (-1.7, 1.3) -- (1.7,2.5) node[right, align=center]{\large Electromagnetic \\  \large emmision};
\end{scope}


\node[inner sep=0pt] (GW2) at (0,-3.1)
    {\includegraphics[trim= 0mm 0mm 0mm 0mm, width=1.1\textwidth]{../GWs/GW2}};




\end{tikzpicture}
\end{figure}
\end{document}
