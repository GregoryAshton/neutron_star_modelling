\documentclass[/home/greg/Thesis/main/main.tex]{subfiles}

\begin{document}

\graphicspath{{/home/greg/Neutron_star_modelling/SpindownRate/PerturbationCalculation/img/}}
\inputpath{/home/greg/Neutron_star_modelling/SpindownRate/PerturbationCalculation/}

\newcommand{\spindown}{\dot{\nu}}
\newcommand{\Aem}{\mathcal{A}_{\mathrm{EM}}}
\newcommand{\Tsd}{\boldsymbol{T}_{\mathrm{s}}}
\newcommand{\wx}{\omega_{\mathrm{x}}}
\newcommand{\wy}{\omega_{\mathrm{y}}}
\newcommand{\wz}{\omega_{\mathrm{z}}}

\section{Approximate analytic spin-down modulation}
The aim of this work is to calculate an approximate form of the spin-down 
modulation capable of exhibiting the key features in the appropriate limit. To 
do this we will first nelect the anomalous torque and work exclusively with the
spin-down torque. In component form, this is given by
\begin{align}
\Tsd = \left(\begin{array}{c}
-\frac{\wx}{2}\left(\cos(2\chi) + 1\right) + \frac{\wz}{2}\sin(2\chi) \\
-\wy \\
\frac{\wx}{2}\sin(2\chi) + \frac{\wz}{2}\left(\cos(2\chi) - 1\right) 
\end{array}\right)
\end{align}
Clearly solving this in general as a source in the Euler rigid body equations is
difficult. In order to get an approximate solution we will define two assumptions
which are acceptable for a small subset of realistic pulsars:
\begin{itemize}
\item We assume that the wobble-angle is small, such that $\omega = \wz + \eta$
      where $\eta \sim \wx, \wy$ and is small.
\item Second, we assume that the magnetic dipole is close to the equator such
      that $\chi \approx \pi/2$. Expanding the trig. functions about this point
      we have
\begin{align}
\sin(2\chi) &\approx -2\left(\chi - \pi/2\right) + \mathcal{O}\left(\left(\chi - \pi/2\right)^{2}\right) \\
\cos(2\chi) &\approx -1 + \mathcal{O}\left(\left(\chi - \pi/2\right)^{2}\right)
\end{align}
\end{itemize}

Under these assumptions, the spin-down torque is:
\begin{align}
\Tsd = \left(\begin{array}{c}
0 \\
0 \\
-\wz 
\end{array}\right) + \mathcal{O}(\eta) + \mathcal{O}(\chi - \pi/2)
\end{align}

Considering the three coupled ODEs for the components of $\spin$, unlike in the
free-precession case where $\dot{\wz} = 0$, we now have:
\begin{align}
\dot{\wz} = -\frac{2R}{3c} \epsA \wz^{3}
\end{align}
This has a physical solution (where $\wz > 0$) given by 
\begin{align}
\wz(t) = \left(\frac{4R}{3c} \epsA t + C_0 \right)^{-1/2}
\end{align}
Solving for $C$ using the initial condition
\begin{align}
\wz(t) = \left(\frac{4R}{3c} \epsA t + \frac{1}{\wz(0)^{2}}\right)^{-1/2}
\label{eqn: wz with torque}
\end{align}

We can demonstrate that this approximation holds by comparing with exact
numerical solutions as shown in figure \ref{fig: wz with torque}. In this plot
we see that the full numerical solution both spins-down and undergoes a
modulation at the precession frequency. By comparing this with the numerical
solution without the anomlous torque, we can understand that the modulations
are a result of the anomalous torque. Without the anomalous torque the numerical
solution agrees well with the analytic solution of equation 
\eqref{eqn: wz with torque}; it should be noted that deviations are found
to exist at the $~10^{-3}$ level.

\FigureAndTable{omegaz}{omegaz}{
         Comparison of three results for the $z$ component of the spin-vector in
         the body frame. In solid black is shown the full numerical solution, the
         dashed black line indicates the numerical solution when the anomalous
         torque component is neglected, in red is shown the analytic result of
         equation \eqref{eqn: wz with torque}}
%\begin{figure}[htb]
%\centering
%\includegraphics[width=0.7\textwidth]{omegaz.pdf}
%\caption{Comparison of three results for the $z$ component of the spin-vector in
%         the body frame. In solid black is shown the full numerical solution, the
%         dashed black line indicates the numerical solution when the anomalous
%         torque component is neglected, in red is shown the analytic result of
%         equation \eqref{eqn: wz with torque}}
%\label{fig: wz with torque}
%\end{figure}

This result holds for any spin-down strength provided the assumptions listed
above are met. However, we can further simplify by working in the weak 
spin-down limit for which $\epsA \ll 1$. Expanding we have
\begin{equation}
\wz(t) \approx \wz(0) + \frac{2R}{3c}\epsA \wz(0)^{3} t + \mathcal{O}(\epsA^{2})
\end{equation}

Now we refer to \citet{Landau1969} where, provided suitable initial conditions
are used, the $\psi$ euler angle is shown to satisfy:
\begin{equation}
\dot{\psi} = -\epsI \wz.
\end{equation}
Plugging in the expanded version of $\wz$ and solving we have
\begin{equation}
\psi(t) = -\epsI\wz(0) t + \frac{R}{3c} \epsA \epsI \omega_{z}(0)^{3} t^{2} + \mathcal{O}(\epsA^{2})
\end{equation}
\begin{figure}[htb]
\centering
\includegraphics[width=0.7\textwidth]{psi.pdf}
\caption{}
\label{}
\end{figure}


%When $\Tsd=0$, the general solution to Eulers rigid body equations are
%\begin{align}
%\wx &= C_2 \cos(\epsI C_1 t) - C_3 \sin(\epsI C_1 t) \\
%\wy &= C_3 \cos(\epsI C_1 t) + C_2 \sin(\epsI C_1 t) \\
%\wz &= C_1 
%\end{align}
%
%We fix the integration constants by specifying the initial coniditions: as in
%the simulation results we begin with the spin vector $\spin$ in the $x-z$ plane
%at an angle $a_{0}$ to the $z$ axis such that $\wz(0) = \omega_{0}\cos(a_{0})$.
%Under the assumptions made above then we obtain a set of solutions for the
%spin-vector in the body frame:
%\begin{align}
%\wx &= \omega_0 \sin(a_0) \cos(\epsI \wz t)  \\
%\wy &= \omega_0 \sin(a_0) \sin(\epsI \wz t) \\
%\wz &= \left(\frac{4R}{3c} \epsA t + \left(\omega_0 \cos(a_0)\right)^{-2}\right)^{-1/2}
%\end{align}

%\subsection{Euler angles}
%Having obtained approximate solutions for the components of the spin-vector in
%the body frame we now need to use these to find approximate solutions for the
%Euler angles $\theta, \phi, \psi$. This can be done by solving the differential
%equations with these approximate solutions
%\begin{align}
%    \dot{\psi} &= -\wz \epsI \\
%    &= -\epsI\left(\frac{4R}{3c} \epsA t + \left(\omega_0 \cos(a_0)\right)^{-2}\right)^{-1/2}
%\end{align}
%Integrating and using the initial condition $\psi(t=0) = \pi/2$ we get
%\begin{align}
%\psi(t) = \frac{-3\epsI c}{2 R \epsA} 
%          \left(\frac{4R}{3c} \epsA t + \left(\omega_0 \cos(a_0)\right)^{-2}\right)^{1/2}
%          + \frac{\pi}{2} + \frac{3 \epsI c}{2 R \epsA \omega_0 \cos(a_0)}
%\end{align}
%Writing this in terms of $\wz$ we have
%\begin{align}
%\psi(t) = \frac{-3\epsI c}{2 R \epsA} 
%          \left(\frac{1}{\wz} - \frac{1}{\wz(t=0)}\right)
%          + \frac{\pi}{2} 
%\end{align}

\biblio
\end{document}

