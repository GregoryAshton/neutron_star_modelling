\documentclass[/home/greg/Thesis/main/main.tex]{subfiles}

\title{Torque Switching}
\author{}

\begin{document}
\graphicspath{{/home/greg/Neutron_star_modelling/TimingNoiseModels/TorqueSwitching/img/}}


\maketitle

\section{Torque switching effects}

\citet{Lyne2010} found strong evidence that a periodic switching in the rate of
spin down $\dot\nu$ induces structure in the residual. Modeling this they found
that for a strictly periodic switching the variations are precisely mirrored in
the residual, if instead a random dither is introduced between switches the
residuals take form very similar to those found in \citet{Hobbs2010}. The model
assumes apriori that the switching occurs with a time scale of several 100
days, the switching itself is motivated by changes in the magnetosphere:
enhances emmison correlates with larger values of $\dot\nu$. A mechanism is
required to keep the magnetosphere in serveral meta-stable states with swithing
occuring on a much shorter time scale than the period between swithces.
\citet{Jones2010} % is that right?
proposed that if the star were to be precessing, its relative orientation to
the angular momentum may bias it towards different magneospheric
configurations. Stars born near the boundary between configurations will then
periodially flip between them on the precession timescale which matches well
with the observed switching times. In this section we aim to model this
proposal by extending our model; this is done by adding a switching into the
torque that depends on the angle $\Psi $ between $\m$ and the angular
momentum. We define the sw

As previously shown this angle is bounded by $|\chi - \theta| < \Psi
< \chi + \theta$ during a precessional cycle, it therefore has an average
value of $\Psi_{\textrm{ave}}\max(\theta, \chi)$; while many different
scenarios could be considered we choose the following: 

\begin{equation}
\boldsymbol{S}(\Psi, \chi, \theta, \upsilon) = \left\{
\begin{array}{ccc}
1 & \textrm{if} & \Psi > \Psi_{\textrm{ave}} \\
1 - \upsilon & \textrm{else} & 
\end{array}•\right. 
\end{equation}
where $\upsilon$ is a number between 0 and 1 parameterising the reduction in torque. 
%
%To illustrate the effect this can have we use a value of $\upsilon=0.8$ 
%
%\begin{figure}[ht]
%\centering
%	\subfloat[ ]{\includegraphics[width=0.55\textwidth]{../../Testing/switching_euler_angles.pdf}} 
%	\subfloat[]{\includegraphics[width=0.55\textwidth]{../../Testing/switching_big_theta.pdf}} \\
%	\subfloat[ ]{\includegraphics[width=0.55\textwidth]{../../Testing/switching_timing_residual.pdf}} 
%	\subfloat[]{\includegraphics[width=0.55\textwidth]{../../Testing/switching_nu_dot.pdf}} \\
%\caption{}
%\label{fig:switching}
%\end{figure}
%
%\FloatBarrier
%\appendix
%\section{Euler angles}
%\begin{equation}
%\left[\begin{smallmatrix}- \sin{\left (\phi \right )} \sin{\left (\psi \right )} \cos{\left (\theta \right )} + \cos{\left (\phi \right )} \cos{\left (\psi \right )} & \sin{\left (\phi \right )} \cos{\left (\psi \right )} + \sin{\left (\psi \right )} \cos{\left (\phi \right )} \cos{\left (\theta \right )} & \sin{\left (\psi \right )} \sin{\left (\theta \right )}\\- \sin{\left (\phi \right )} \cos{\left (\psi \right )} \cos{\left (\theta \right )} - \sin{\left (\psi \right )} \cos{\left (\phi \right )} & - \sin{\left (\phi \right )} \sin{\left (\psi \right )} + \cos{\left (\phi \right )} \cos{\left (\psi \right )} \cos{\left (\theta \right )} & \sin{\left (\theta \right )} \cos{\left (\psi \right )}\\\sin{\left (\phi \right )} \sin{\left (\theta \right )} & - \sin{\left (\theta \right )} \cos{\left (\phi \right )} & \cos{\left (\theta \right )}\end{smallmatrix}\right]
%\label{eqn:rotation matrix}
%\end{equation}•
%
%
%\section{Catastrophic cancellation}
%The ODEs defined in \eqref{eqn:ODEs} suffer a 


\biblio
\end{document}

